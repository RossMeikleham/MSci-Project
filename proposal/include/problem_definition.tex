\section{Problem Description}

We want to exploit the parallelism of sequential scientific Fortran code without
having to heavily modify the code itself. There are a multitude of ways to do this but
we focus on using FPGAs (Field Programmable Gate Arrays). 

A FPGA is an integrated circuit that contains an array of programmable logic blocks and interconnects which
wire the blocks together. These logic blocks operate in parallel and are programmed using a Hardware Description Language(HDL)
such as Verilog or VHDL. Currently designing programs in a HDL is a tedious and error prone process, especially
as programs start to become more complex \cite{complex}.
However, code that is highly parallelizable can result in significant performance gains when synthesized to
run on a FPGA.

The main focus in research on synthesizing sequential code to HDL has been on creating compilers which generate
HDL code from C, C++, and MATLAB rather than Fortran. The goals of many of these compilers are focused on reducing programming 
time and software maintenance issues rather than optimising for performance. Often, optimizing
for space and architecture optimizations requires modifying of the original source code, and are not
provided by the compilers themselves \cite{hlstools}.

There can be a multitude of ways to parallelize certain sequential code, for example consider the Fortran95 program in
listing~\ref{f95} which adds 1 to each element of a 12 element array.

\lstinputlisting[caption={F95 Loop}, label=f95, xleftmargin=4.0ex]{code/loop.f95}

Considering that each array element is independent of one another during the adding operation
it's entirely possible to perform the addition of each element in parallel. We can tune how much
we parallelize the array addition using factors of the size of the array. In this case the factors of 12 are 1, 2, 3, 4, 6, and 12.

For each factor \textbf{f} of 12 there exists a factor \textbf{f'} such that \textbf{f * f'} = 12. Let \textbf{f} be the number of 
circuits and \textbf{f'} be the number of additions performed by each component, increasing \textbf{f} increases the degree of 
parallelization but also increases the total area used on a FPGA board. 

Given a cost model and a program we want to optimise a program with respect to the given cost model, this
involves generating equivalent variants of the given program, and then selecting the program which
gives the best result from the given model to synthesize.
