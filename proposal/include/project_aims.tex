\section{Project Aims}
The main aim of this project is to design a framework which acts as an intermediate stage in HDL synthesis from Fortran code onto
an FPGA. We restrict the framework to using a pure functional approach; representing a program as vector structures operated
on with a combination of element processing functions with map, fold, and zip operations.

This approach allows us to perform type transformations on the computations to tune the amount of parallelism
whilst still preserving the results of the computations. 

We aim to model the framework as an EDSL (Embedded Domain Specific Language); this saves the tedious work
of having to impelent a custom DSL (Domain Specific Language) for this specific purpose, allowing us to use the 
syntax and of the language used to implement the framework. This also has the added benefit of being able to interpret 
these programs on a typical desktop computing platform which can be useful for testing purposes before transforming into a 
HDL. 

We also aim to implement a rudementary cost model which should be able to be replaced with more complex
cost models in the future. Each given function has a given performance and space cost, represented as integers. 
The lower the integer the lower the performance cost, and vice versa. The same goes for the space cost. 
For example given a function with performance cost \textbf{p}, and space cost \textbf{s} which is being mapped on over
a 1 dimensional vector with \textbf{n} elements; performing the map entirely in parallel would incur a performance
cost of \textbf{p}, and space cost of \textbf{n * s}. Performing the map sequentially would incur a performance cost of
\textbf{n * p}, and a space cost of \textbf{s}. A higher degree of paralleisation increases the space cost but reduces
the performance cost, a lower degree of paralleisation reduces the space cost but increases the 
performance cost.

Another aim is to also be able to create an interface for converting the optimal program for the given cost model 
into another format, such as a possible bytecode format which is then itself processed and converted to a HDL.

The focus of this project isn't on translating Fortran code into the EDSL, or directly translating the optimal program 
into a HDL.
