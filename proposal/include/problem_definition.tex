\section{Problem Description}

We want to exploit the parallelism of scientific Fortran code without
having to modify the code itself. There are a multitiude of ways to do this but
we want to focus on using FPGAs (Field Programmable Gate Arrays). 

An FPGA is an integrated circuit that contains an array of programmable logic blocks and interconnects which
wire the blocks together. These logic blocks operate in parallel and are programmed using a Hardware Description Language(HDL)
such as Verilog or VHDL. Currently designing programs in a HDL is a tedious and error prone process. 
However, code that is highly parallelisable can result in significant performance gains when converted to
run on an FPGA.

The main focus in research on synthesising sequential code to HDL has been on creating compilers which generate
HDL code from C rather than Fortran. The goals of many of these compilers are focused on reducing programming 
time and software maintenance issues rather than optimising for performance. The compilers that do
focus on paralellisation essentially parallelise as much as possible, this is not always ideal
as there may be other requirements such as the overall size of the circuits generated.  

There is a multitude of ways to parallelise certain code, for example consider the following Fortran95 program
which adds 1 to each element of a 12 element array.

\lstinputlisting{code/loop.f95}

Considering that each array element is independent of one another during the adding operation
it's entirely possible to perfom the addition of each element in parallel. We can tune how much
we parallelise the array additition using factors of the size of the array. In this case the factors of 12 are 1, 2, 3, 4, 6, and 12.

For each factor \textbf{f} of 12 there exists a factor \textbf{f'} such that \textbf{f * f'} = 12. Let \textbf{f} be the number of 
circuits and \textbf{f'} be the number of additions performed by each circuit, increasing \textbf{f} increases the degree of 
paralleisiation but also increases the total number of circuits required on the board to support this. 

Given a cost model and a program we want to optimise the program with respect to the given cost model, this
involves generating equivalent variants of the given program.
